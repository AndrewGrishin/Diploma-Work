\subsection{Развитый рынок (США)}
В данном разделе задаемся вопросом об особенностях рассматриваемых нами рынков. Несложно предположить, что понимание природы рынков позволит сделать и некоторые выводы о динамике цен акций.

Начнем с разделения понятий «развитые рынки» и «развитые страны»: хорошо функционирующий финансовый и, как его составляющая, фондовый рынки не подразумевают наличие статуса развитой страны, который учитывает развитие человеческого капитала, а также статуса развитой экономики. Отлично иллюстрирует это пример Китая, однозначно развитой экономики, однако имеющей статус развивающихся рынков. Определения для развитых и развивающихся фондовых проще составить, ссылаясь на их характерные свойства. Заметим, что развитые рынки были созданы относительно давно и имеют длинную историю становления, поэтому чаще находятся в странах с высоким уровнем развития. Стоит заметить, что и развитость страны также косвенно связана с наличием эффективно функционирующего рынка ценных бумаг, благодаря ему высокотехнологичные отрасли могут получать достаточно инвестиций для совершения технологических прорывов, в прочем, как и другие агенты реального сектора, взаимодействие и перераспределение капитала выходит на качественно новый уровень. Благодаря имеющемуся опыту становления развитые рынки характеризуются стабильностью и высокой капитализацией (большой суммарной стоимостью представленных на нем ценных бумаг). Логично предполагаем, что если оба показатели низки, то фондовый рынок находится на стадии развития и имеет характерные черты, связанные с инфраструктурой, законодательством, информационной доступностью и точностью, а также динамикой торгов. Даже активность, связанная с краткосрочными вложениями в ценные бумаги, хоть и не является прямой инвестицией в развитие экономики, но приводят к возрастанию стоимости компаний на национальном рынке и способствуют инвестиционной привлекательности рынка в целом \cite{kulakov2016economic}, создавая условия для работы сбережений рыночных агентов. Сбережения таким образом превращаются в инвестиции.

Все это успешно реализуется на развитых рынках. Незначительные колебания цен способствуют сокращению неопределенности и подразумевает небольшую волатильность, а развитая система регулирующих норм страхует инвесторов от чрезвычайных ситуаций или нечестного поведения агентов. Надежность инструментов и продолжительный опыт их использования позволяют уходить от спот-рынка к производным инструментам. Также характерными чертами являются большие объемы операций, высокая капитализация отдельных компаний и как следствие рынка. Так, рынок Китая находится на втором месте в рейтинге капитализации \cite{statista2023distribution} после Соединенных Штатов, однако демонстрирует более чем в 3 раза меньшие значения. Подробнее о капитализации рынков разного типа ведем разговор далее. \textit{Отсутствие дискриминации миноритарных инвесторов, доступ иностранных рыночных агентов}, а также отсутствие экстремальных транзакционных издержек свидетельствует о развитости рынка. 

Классическим примером такого рынка является самый крупный фондовый рынок мира, принадлежащий Соединенным штатам Америки, что еще раз подчеркивает взаимную обусловленность фондового рынка и экономики в целом. Отличительной особенностью и по сути главной характеристикой служит разнообразие как таковое: разнообразие акций, представленных отраслей, используемых инструментов, бирж, участников торгов. Биржи США аккумулируют акции крупнейших компаний мира. Здесь редки случаи дефицита спроса или предложения: рынок характеризуется высокой ликвидностью. Рынок США также имеет одну из самых надежных систем регулирования в мире, что гарантирует участникам рынка защищенность и безопасность. Самый первый законодательный акт был заключен еще в 1934 году (Закон о ценных бумагах).

Все ключевые работы в отношении финансовых рынков были написаны экономистами из США. Будь то Гарри Марковиц, разработавший портфельную теорию и теорию эффективных рынков, или Уильям Шарп, который известен в первую очередь как один из создателей CAMP, или Франко Модильяни и Мертон Миллер и так далее. Их исследовательские работы помогли сформировать современную систему финансового управления и инвестирования на фондовов рынке США. 

Немаловажным фактом является повышенная информативность, прозрачность информации создает благоприятные условия для отслеживания всего рынка. Также большое количество профессиональных участников, что также добавляет надежности и обеспечивает высокий уровень экспертизы. 
