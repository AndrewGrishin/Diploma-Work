\section{Заключение}
Подводя итог всего настоящего исследования, заново формализуем проблему, обсуждаем выдвинутую гипотезу, а также рассматриваем анализируемые данные, полученные результаты, коротко говорим о выводах, и проверяем --- выполнены ли цели и соответствующие им задачи. В конце указываем задел на будущее относительно дальнейших исследований.

В современном мире крайне важно уметь <<предвидеть>> то, что будет дальше. Необходимости этого возникает в силу развития как технологий, так и финансовой сферы деятельности человека. Новые знания равны новым трудностям, следовательно для обеспечения собственного благосостояния человек (любой --- не обязательно связанный с биржей) должен уметь зарабатывать деньги. Главным объектом анализа настоящего исследования является цена открытия и доходность компаний, котирующихся на Нью-Йоркской и Шанхайской биржах.

Важно понимать, что выбраны 15 компаний для более точного результата без смещения на конкретную отрасль. Срок предсказания --- один рабочий день биржи, то есть проводится проверка не способности долгосрочного прогнозирования, а наоборот --- краткосрочного. Анализируемые модели --- это популярные эконометрические, статистические, а также --- достаточно простые нейросетевые.

Целями настоящего исследования является помощь как трейдерам, так и кому бы то ни было, связанному с биржей, в оперативном формировании точного ответа на вопрос: <<Buy, Hold or Sell>>. Что в итоге сделает биржевые сделки менее рискованными, из чего получается планирование на более длительный промежуток времени в силу финансовой определенности. В итоге людям, например, не имеющим прямого отношения к бирже (или являющимся неопытными в этом вопросе), станет проще адаптироваться к фондовым рынкам, так как они (люди) получат возможность достаточно точно прогнозировать необходимые для принятия решения показатели.

Задачами же моего исследования является проведение последовательного сравнительного анализа рассматриваемых эконометрических, статистических и нейросетевых моделей в задаче прогнозирования доходностей и --- как будет показано далее --- цен акций 15 американских и 15 китайских компаний сроком на 1 рабочий день биржи. Для наиболее точного результата без смещения на особенности рассматриваемой индустрии, обсуждается именно 15 компаний, взятых из разных сфер производства и услуг.

Гипотезами, на которых строится все исследование, являются: Гипотеза Рыночная Эффективность Юджина Фамы, статья о которой выпущена в 1970 году, а также Гипотеза Рыночной Фрактальности, сформулированная Бенуа Мандельбротом и опубликованная им в 2006 году.

В противоположность ко всем вышесказанным идет гипотеза <<рыночной неэффективности>> Мартина Севелла, опубликованная в 2011 году, по которой ГРЭ является лучшим, что может быть на сегодняшний день, однако ложным в силу наличия большого количества исследований это показывающих. В своей работе Мартин Севелл анализирует всю доступную информацию, посвященную становлению ГРЭ в том виде, в котором она нам известна.

В настоящем исследовании применяются модели как эконометрического характера, основанные на корреляции, статистические модели, вообще ничего не <<думающие>> об имеющихся данных, а также --- нейросетевые модели, держащие в своей основе сложную самообучающуюся функцию, и называемую нейронной сетью. 

В эконометрической области выбраны наиболее популярные ARIMA и ее фрактальное дополнение ARFIMA, GARCH и аналогично FIGARCH.

Из статистики тут EWMA (экспоненциальная средняя), а также SSA (в народе более известный как метод <<Гусеница>>), представляющий из себя сингулярное спектральное разложение имеющегося ряда.

От нейросетей взят наиболее простой набор моделей (полносвязные, рекуррентные, а также wavelet сеть, заимствующая подход у Wavelet анализа), чтобы показать, как ошибка предсказаний достаточно сложных Эконометрических и Статистических моделей отличается от достаточно простых Нейросетевых. В качестве очистителя от шума имеющихся данных используется алгоритм MSSA.

Наиболее популярные на данный момент в сфере обработки естественного языка <<трансформеры>> не используются. Причина --- отсутствие контекста для анализа.

Топологическая функция для ранжирования полученных моделей --- WAPE (Weighted Average Percentage Error), позволяющая видеть порядок отклонения предсказанного значения от его реальной величины. Все, сказанное далее основывается на имеющихся данных и не претендует на $100\%$ точность.

В полученных результатах видим, что MLP + MSSA --- фактически лучшая модель для работы с ценам на развитом и развивающемся рынках. При это очень важна оговорка, что WN + MSSA --- аналогично своему оппоненту не теряет оборотов и также показывает (путь и не для всех компаний) хороший результат, который нельзя игнорировать. Подобная ситуация говорит о наличии как на развитом, так и на развивающемся рынка доминантных частотных паттернов, поддающихся анализу посредством Wavelet преобразования.

Для доходностей же --- лучшей моделью является RNN + MSSA (для развитых) и RNN/RNN + MSSA (для развивающихся). Однако в силу достаточно большой погрешности нельзя однозначно говорить именно о прогнозе доходности. То есть, конечно, лидер выявлен, но полученных от него результат не позволяет применять данные алгоритм на практике в силу большого финансового риска. Таким образом, необходима дальнейшая разработка моделей, отмечающих непосредственно за доходность.

Также отмечаем, что алгоритм очистки данных от шума MSSA в среднем дает хороший прирост к точности прогноза ко всем моделям. Исключением в этом случае иногда является модель WN + MSSA, для которой в силу архитектуры избавление от шума может только усугубить ситуацию.

Эконометрические методы плохо подходят как для прогнозирования цен, так и доходностей, а вот статистические (EWMA и SSA) --- показывают сравнимый с нейросетевым результат и при этом не требуют таких больших вычислительных мощности, как нейронные сети при обучении. Получаем компромисс: если нужно более точно (и при этом есть время) --- нейронные сети, иначе (нужно максимально быстро) --- статистические модели типа EWMA и SSA.

В итоге основные задачи исследования выполнены: то есть выбраны лучшие для того или иного направления исследования --- будь то цены или доходности --- выбраны. Финальная таблица предоставлена и проанализирована. Однако главной целью настоящего исследования было --- помочь трейдерам в принятии более достаточно неоднозначного биржевого решения относительно вопроса <<Buy, hold or sell>>. Конкретно этот пункт возможно выполнить только при условии развития и имплементации выбранного в настоящей работе алгоритма для биржевых процессов. Только после прохождения данного <<боевого крещения>> можно однозначно сказать --- выполнена ли эта цель или нет.

Относительно помощи людям в вопросах к биржевым процессам, уверен цель достигнута, так как, углубляя свои знания в математическом аппарате, человек невольно начинает сам пытаться выявлять зависимости. Таким образом, нейронные сети --- лишь очередной толчок для человека к развитию. Иными словами, пока у человека есть желание развиваться, он будет развиваться и дальше. В итоге --- объединение человека и алгоритма является продолжением существования обоих как в цифровой среде (будь то как просто интернет, так и фондовые рынки), так и в реальной жизни. Ведь доверие компьютеру --- достаточно сложный шаг, доступный пока что далеко не каждому.

В заключение, выражаю надежду на дальнейшее развитие темы, изложенной в настоящем исследовании, так как выдвинутая автором работы гипотеза крайне интересна как и с научной точки зрения, так и с практической. Ведь, что может быть лучше, чем хорошие данные, правильная их интерпретация и качественный алгоритм, моделирующий их?

