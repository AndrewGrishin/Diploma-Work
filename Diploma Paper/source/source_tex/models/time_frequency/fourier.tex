\subsubsubsection{Анализ Фурье} \label{link::fourier_analysis}
\\\\
\indent По изначальному подходу, сигнал $f(t)$ - это некоторая функция, то есть ее можно пытаться чем-то приближать. Конечно, далеко не всякую функцию можно аппроксимировать, однако пока что оставляем математическую строгость за кадром и просто пробуем. Но в таком случае из чего-то сложного как исходный сигнал, получаем что-то более простое, но поддающееся анализу, значит, мы раскладываем имеющуюся $f(t)$ на простые составляющие. Замечаем, что в первом приближении для этого должны быть выбраны простые функции, которые легко поддаются анализу. В таком случае, задаемся вопросом~\textbf{Q}: Можно ли как-то представить функцию вида $f(t)$ (исследуемый непрерывный сигнал) в виде конечного или бесконечного наложения синусоид (под синусоидой понимается как косинус, так и синус). \textbf{A}:~1)~Замечаем, что существует неточность: почему именно синусоиды? \textbf{Q}:~Почему нельзя взять иной набор функций, на которые раскладывается сигнал? \textbf{A}: Любой набор взять нельзя, однако при определенных условиях подобный подход приводит к рассматриваемому далее Wavelet анализу (\myref{link::wavelet_analysis}). На данный момент останавливаем рассуждения на  упоминании синусоид. Изначально было отмечено \cite{mipt2021string}, что струна - это тонкая и гибкая нить, способная совершать колебания при условии фиксированных концов как раз на основе синусоидальных форм. Причем по всей длине струны всегда умещается конечное количество волн. Это обеспечивается условием закрепленности струны на обоих концах. 

Самая простая форма колебания струны называется "гармоникой". Отсюда и происходит второе название анализа Фурье - гармонический анализ. Таким образом, формализуя исходную задачу, получаем выражение (\ref{equation::fourier_approximation}). Отмечаем, что факт того, как данное выражение изначально получено находится за областью, исследуемой в настоящей работе. Следовательно, предполагаем, что подобная форма разложения в ряд Фурье была <<угадана>>. 
\begin{equation} \label{equation::fourier_approximation}
	f(t) = \frac{a_0}{2} + \sum_{k = 1}^{\infty} a_k\cos(k \omega t) + b_k\sin(k \omega t)
\end{equation}
\indent Однако, несмотря на вольность предположений, далее совершенно строго находим для данного разложения все необходимые коэффициенты. Более того, отмечаем, что пока не было введено никаких предпосылок относительно $f(t)$, но далее они потребуются.

Пусть исходный сигнал $f(t)$ является периодическим с периодом $T$, тогда, основываясь на знаниях из Линейной Алгебры об ортогональных векторах, по аналогии вводим операцию <<измеряющую степень>> ортогональности функций~\cite{penn2016orthogonal}. В итоге получаем:
\begin{equation}
	p = \int_0^T g(t) f(t) \; \text{d}t
\end{equation}
\indent Далее вычисляем данную меру для выбранных в качестве базисных функций $\cos(\cdot)$ и $\sin(\cdot)$. Если $p \ne 0$, то вся проводимая операция не может быть названа разложением по базисным функциям. В силу того, что под <<синусоидой>> подразумевается как $\sin(\cdot)$, так и $\cos(\cdot)$, вычисляем меру $p$ для каждого из четырех возможных вариантов их комбинаций.
\begin{enumerate}
	\item $\sin(k\omega t), \sin(l \omega t): k, l \in \N, \omega \in \R^+$.
	\begin{equation}
		p = \int_{0}^T \sin(k\omega t)\sin(l \omega t) \; dt =
		\left\{
		\begin{array}{rl}
			0 & \text{, } k \ne l\\
			T / 2 & \text{, } k = l
		\end{array}
		\right.
	\end{equation}
	
	\item $\cos(k\omega t), \cos(l \omega t): k, l \in \N, \omega \in \R^+$.
	\begin{equation}
		p = \int_{0}^T \cos(k\omega t)\cos(l \omega t) \; dt =
		\left\{
		\begin{array}{rl}
			0 & \text{, } k \ne l\\
			T / 2 & \text{, } k = l
		\end{array}
		\right.
	\end{equation}
	
	\item $\cos(k\omega t), \sin(l \omega t): k, l \in \N, \omega \in \R^+$.
	\begin{equation}
		p = \int_{0}^T \cos(k\omega t)\sin(l \omega t) \; dt = 0
	\end{equation}
\end{enumerate}
Тогда, опираясь на  выражение (\ref{equation::fourier_approximation}) и умножение его обеих частей на $\sin(k\omega t)$ и $\cos(l \omega t)$, получаем формулы для коэффициентов. Рассматриваем только для $\sin(\cdot)$, так как аналогичным образом получается для $\cos(\cdot)$.
\begin{equation}
	\int_{0}^T f(t) \sin(p \omega t) dt = \int_0^T \sin(p \omega t) \left(\sum_{k = 1}^{\infty} a_k\cos(k \omega t) + b_k\sin(k \omega t) \right) dt
\end{equation}
Однако изменять порядок интегрирования и суммирования, при условии бесконечной суммы, можно только в случае равномерной сходимости ряда. Для настоящего ряда проблем в сходимости не наблюдается \cite{teljacovski2001convergence}, \cite{mipt2004fourier}. Преобразуем выражение, приняв во внимание факт ортогональности функций, описанный выше:
\begin{equation}
	\begin{split}
		\int_{0}^T f(t) \sin(p \omega t) \; dt & = \sum_{k = 1}^{\infty} b_k \int_0^T \sin(k \omega t)\sin(p \omega t) \; dt = T / 2  \cdot b_p \\
		b_k & = \frac{2}{T}  \int_{0}^T f(t) \sin(k \omega t) \; dt 
	\end{split}
\end{equation}
Аналогичным образом получаем выражение для коэффициентов $a_k$:
\begin{equation}
	a_k = \frac{2}{T}  \int_{0}^T f(t) \cos(k \omega t) \; dt 
\end{equation}
В более удобной форме вся задача приближения выглядит:
\begin{equation} \label{eq::fourier_task}
	\begin{split}
		f(t) & = \frac{a_0}{2} + \sum_{k = 1}^{\infty} a_k\cos(k \omega t) + b_k\sin(k \omega t)\\
		& = \frac{a_0}{2} + \sum_{k = 1}^{\infty} A_k \cos(k \omega t - \phi_k)\\
		A_k & = \sqrt{a_k^2 + b_k^2}\\
		\phi_k & = \arctan(b_k / a_k)\\
	\end{split}
\end{equation}
Где $A_k$ - амплитуда сигнала, а набор амплитуд $\left\{A_k\right\}_{k = 1}^\infty$ - спектр сигнала, $\phi_k$ - фаза сигнала. Переход к подобной записи как после 2-ого знака равенства сделан для того, чтобы было интуитивно удобнее характеризовать величину посредством не синуса и косинуса, а амплитуды и фазы в конкретный момент времени. Однако (\ref{eq::fourier_task}) можно еще больше упростить, перейдя в комплексную плоскость и воспользовавшись тождеством Эйлера:
\begin{equation}
	\exp(i k \omega t) = \cos(k \omega t) + i \sin(k \omega t)
\end{equation}
Отсюда получаем выражения для $\cos(k \omega t)$ и $\sin(k \omega t)$, после чего переписываем не все приближение, а только конкретный коэффициент $k$ (это сделано для удобства восприятия):
\begin{equation}
	\begin{split}
		a_k \cdot \frac{\exp(i k \omega t) + \exp(-i k \omega t)}{2} & + b_k \cdot \frac{\exp(i k \omega t) - \exp(-i k \omega t)}{2i}\\
		\underbrace{\left( \frac{a_k}{2} - \frac{b_k}{2} i \right) \exp(i k \omega t)}_{\text{<<Положительная>> часть}} & + \underbrace{\left( \frac{a_k}{2} + \frac{b_k}{2} i \right) \exp(-i k \omega t)}_{\text{<<Отрицательная>> часть}}
	\end{split}
\end{equation}
Отсюда получаем:
\begin{equation}
	c_k = \left\{
	\begin{array}{rl}
		a_k / 2 - i \cdot b_k/ 2  & \text{ ; } k \ge 0\\ 
		a_k / 2 + i \cdot b_k/ 2  & \text{ ; } k < 0\\ 
	\end{array}
	\right.
\end{equation}
В конечно виде (\ref{eq::fourier_task}) записывается уже как:
\begin{equation}
	f(t) = \sum_{k = -\infty}^{\infty} c_k \exp(i k \omega t)
\end{equation}
Однако тут интерес представляют сами коэффициенты $c_k$, так как именно они и являются коэффициентами Фурье при проецировании на базисные функции. Вспоминаем, что ранее делалась предпосылка, что сигнал $f(t)$ является периодическим. \textbf{Q}: Как быть в противном случае? \textbf{A}: Необходимо сам период устремить к $\infty$. Но прежде подставляем вместо некоторой частоты $\omega$ конкретное значение: $\omega = k \Delta \omega$, где $\Delta \omega = \pi / T$. Следовательно, $c_k = \langle f, \psi_k\rangle / (2 \pi)$, где $\psi_k \equiv \exp(i k t)$. Значит:
\begin{equation}
	\begin{split}
		f(t) & = \lim_{\Delta \omega \to 0} \sum_{k = -\infty}^{\infty} \frac{\Delta \omega}{2 \pi} \left[\int_{-\pi / \Delta \omega}^{\pi / \Delta \omega} f(\xi) \exp(-i k \Delta \omega \xi) \; d \xi\right] \exp(i k \Delta \omega)\\
	\end{split}	
\end{equation}
Но подобное выражение, путем вычисления предела, превращается в:
\begin{equation}
	f(t) = \int_{-\infty}^{\infty} \frac{1}{2\pi} \underbrace{\left\{\int_{-\infty}^{\infty} f(\xi) \exp(-i \omega \xi) \; d \xi \right\}}_{\hat{f}(\omega)} \exp(i \omega t) \; d \omega
\end{equation}