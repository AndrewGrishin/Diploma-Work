\subsubsubsection{Быстрое преобразование Фурье (FFT)} \label{link::fft}
\\\\
\indent Задаемся следующим вопросом: \textbf{Q}: Как вычислить FT (Fourier Transform) на компьютере? \textbf{A}: Для этого в 1965 году американские математики John Tukey и James Cooley разработали алгоритм быстрого преобразования Фурье \cite{cooley1965algorithm}, который в отличие от своего оригинального варианта выполняется асимптотически за $\mathcal{O}(n \log n)$ вместо $\mathcal{O}(n^2)$, что позволяет крайне быстро (почти линейно по количеству операций) обрабатывать большие массивы данных. Более подробно о схеме работы и его принципе рассказано в \cite{brunton2022data}. Однако интересно, что своей эффективностью алгоритм обязан простой перестановке элементов с четными и нечетными индексами (перегруппировка), что позволяет рекурсивно расщеплять матрицу весов, в которой и зашиты коэффициенты разложения Фурье. Повторяя данный процесс итеративно (и при условии того, что $n = 2^k: k \in \N$), получаем высокую вычислительную скорость. Особенность заключается в том, что исходная матрица весов (коэффициентов Фурье) является симметричной и более того представляет из себя матрицу Вандермонда, только тут она еще, исходя из слов о симметричности, является квадратной. 