\subsubsection{Neural Networks} \label{link::neural_networks}
Наиболее современным методом моделирования временных рядов как ранее описанных в настоящей работе, так и далее используемых в исследовании, является применение нейросетевого подхода, в основе которого лежит <<самообучающаяся>> функция, называемая нейронной сетью. 

Впервые предложенный Warren S. McCulloch и Walter Pitts \cite{mcculloch1943logical}, подход проведения вычислений сродни поведению нейрона стал набирать популярность как в областях применения: регрессия, классификация, кластеризация, так и в областях построения нейронных сетей. В 1961 году за счет трудов Frank Rosenblatt'а были изобретены многослойные персептроны \cite{rosenblatt1961principles}, за этим в 1980-х последовала разработка сверточных нейронных сетей, максимально эффективно --- на тот момент --- работающих с изображениями \cite{lecun1989backpropagation, lecun2015deep, lecun1989generalization}. После чего, когда речь зашла об обработке естественного языка были предложены рекуррентные нейронные сети, основанные не итеративном <<сборе информации>> о предоставляемых текстах \cite{hochreiter1997long, rumelhart1986learning}. 

В настоящий момент нейронные сети спокойно работают с видео, звуком, а также --- при определенных затратах вычислительных мощностей --- генерируют картинки. Однако State Of The Art (SOTA) почти во всех современных моделях глубокого обучения выступают трансформеры, в основе которых лежит механизм самовнимания \cite{attention_transformers}, позволяющий этим тяжелым сетям обрабатывать огромное количество информации, а что самое главное --- запоминать и вычленять. Однако в области прогнозирования временных рядом трансформеры пока что не занимают лидирующих позиций \cite{transformers_are_useless_for_TSF}.  Поэтому, так как в ценах/доходностях акций, конечно, содержится информация, однако она не носит лингвистический характер, которая позволяла бы применять подход контекстного анализа --- предсказываем то или иное слово в тесте, исходя из его контекста \cite{word2vec_2013}. Именно по этой причине в настоящем исследовании не рассматривается применение данный алгоритмов глубокого обучения при построении прогноза временного ряда.

\subsubsubsection{Multilayer Perceptron}
\\\\
\indent Ведя разговор о нейронных сетях, нельзя не затронуть сам способ построения подобных моделей. Впервые предложенный Frank Rosenblatt'ом \cite{rosenblatt1961principles}  в 1961 году принцип построения данных моделей, основанный на подобии того, как работает нейрон в мозгу человека, достаточно сильно изменялся, однако идея осталась прежней. По определению <<нейронная сеть>> --- последовательное преобразование признакового пространства. Соответственно формализованный вид данной операции имеет запись:
\begin{equation}
	a(X) = \phi_k(W_k \cdot \ldots \cdot \phi_2(W_2 \cdot \phi_1(W_1 X)))
\end{equation}
Где $W_j: j = \overline{1, k}$ --- некоторая матрица, в общем случае размер ее указать не удается, так как на каждом этапе вычисления получается переход в новое пространство с новой размерностью, $X \in \R^{n \times m}$ --- матрица объект-признак, используемая в качестве входных данных, а $\phi_j : j = \overline{1, k}$ --- используемая на конкретном слое функция активации. Сама же задача, появляющаяся перед исследователем имеет вид:
\begin{equation}
	\frac{1}{m} \sum_{j = 1}^m L(a(x_j | w), y_j) - R(w) \cdot \lambda \to \min_{w}
\end{equation}
Перед нами в общем случае невыпуклая огромная сумма, где $L(\cdot)$ --- функционал потерь, $w = \left\{W_1, \ldots, W_k\right\}$ --- не вектор, а просто набор данных, а $R(\cdot)$ --- функция регуляризации набора весов $w$, чтобы уменьшить вероятность переобучения. Соответственно решение аналитические получить невозможно, однако существуют различного рода алгоритмы оптимизации, позволяющие достигать локального/глобального минимума итеративно. В общем случае интуиция за достижением минимума выглядит так:
\begin{equation}
	w^{t + 1} = w^t - \eta^t \cdot \nabla_w \left[L(a(x_j | w), y_j) - R(w) \cdot \lambda\right]
\end{equation}
Наиболее популярными в этой области являются Adam \cite{kingma2014adam}, AdamW \cite{bock2018improvement}, LBFGS \cite{liu1989limited}, а также недавно (2023 год) разработанный Lions \cite{chen2023symbolic}. В настоящем исследовании применяются исключительно LBFGS (для обучения MLP), а также --- в случаях, когда LBFGS дает неудовлетворительный результат --- AdamW как наиболее распространенный (универсальный) его заменитель.

Сама идея алгоритма обучения завязана на применение теоремы о дифференцировании сложной функции, так как $a(\cdot)$ именно ей и является. Таким образом, для вычисления такого количества производных, используется алгоритм под названием BackPropagation \cite{linnainmaa1970representation}. Принцип его работы заключается в дифференцировании графа вычислений, олицетворяющего исходную нейронную сеть. В формальном виде:
\begin{equation}
	\frac{\partial L}{\partial w} = \left. \frac{\partial f_1}{\partial w} \right\rvert_w \left. \frac{\partial f_2}{\partial f_1} \right\rvert_{f_1(w)} \cdot \ldots \cdot \left. \frac{\partial f_k}{\partial f_{k - 1}} \right\rvert_{f_{k - 1}(\cdot)} \left. \frac{\partial L}{\partial f_k} \right\rvert_{f_k\left(f_{k - 1}(\cdot)\right)} = \nabla_w L
\end{equation}
Интересно отметить, что существует теорема, гласящая, что имея однослойную нейронную сеть, можно с любой точностью приблизить абсолютно любую непрерывную функцию многих переменных \cite{cybenko1989approximation}.

В качестве примера рассматриваем способность к прогнозированию на 1 рабочий день биржи модели MLP для цен акций открытия компании Ford, начиная с даты выхода компании на IPO, а заканчивая 13/12/2022 годом.
\begin{figure}[H]
	\centering
	\includegraphics[width=17cm]{returns pictures/ford_prices_returns.png}
	\caption{График цены открытия и доходности Ford (IPO - 2022)}
	\label{fig::ford_prices_returns}
\end{figure}
Далее смотрим на показатели функции потерь для обучающей выборки и валидационной по мере обучения самой модели. В качестве функции потерь в данном случае используется уже ранее обсуждавшаяся MSE, а в качестве метрики использовалась WAPE (Weighted Average Percentage Error):
\begin{figure}[H]
	\centering
	\includegraphics[width=17cm]{nn/mlp/ford_train_val_results.png}
	\caption{График MSE для модели MLP}
	\label{fig::ford_train_val_results}
\end{figure}
Но также интересно, как менялось значение лучшей валидационной метрики, ведь именно она является критерием сохранения весов модели. Принцип прост: если меньше максимального, сохраняй модель.
\begin{figure}[H]
	\centering
	\includegraphics[width=17cm]{nn/mlp/ford_best_metric_results.png}
	\caption{График изменения лучшей валидационной метрики: WAPE$(y, \hat{y}) = \sum_{t = 1}^n |y_t - \hat{y}_t| / \sum_{t = 1}^n |y_t|$}
	\label{fig::ford_train_best_metric_results}
\end{figure}
Финальным этапом является построение прогноза и подсчета получившейся ошибки в терминах процентного отклонения WAPE и аналогично ранее указанной RMSE.
\begin{figure}[H]
	\centering
	\includegraphics[width=17cm]{nn/mlp/ford_test_results.png}
	\caption{График реальных и предсказанных значений цен акций Ford (USD)}
	\label{fig::ford_test_results}
\end{figure}
Очевидно, что методика неплохо работает, так как полученная метрика без наличия тюнинга (подбора гиперпараметров) составляет $\pm 1.29$ USD, что достаточно мало, но существенно в случае большого количества акций, с которым совершается операция. Однако остается провести аналогичное исследование для доходностей, чтобы в полной мере понять способность модели к из прогнозированию.
\begin{figure}[H]
	\centering
	\includegraphics[width=17cm]{nn/mlp/ford_train_val_returns_results.png}
	\caption{График MSE для модели MLP (доходности \%)}
	\label{fig::ford_train_val_returns_results}
\end{figure}
Поведение достаточно схожее с моделью на ценах, то есть обучение выполняется достаточно успешно. Однако пока что это ничего не говорит о самом поведении модели на тестовой выборке, ведь данная кривая символизирует лишь успешную сходимость к минимуму. То есть даже, если ошибка мала, результат может быть неудовлетворительным, из-за необходимости обоснования поведения обученной модели, исходя из логической составляющей, прикладываемой к реальности. Иными словами, полученные результаты должны согласовываться с реальностью, даже если показатель ошибки минимален.
\begin{figure}[H]
	\centering
	\includegraphics[width=17cm]{nn/mlp/ford_best_metric_returns_results.png}
	\caption{График изменения лучшей валидационной метрики: WAPE$(y, \hat{y}) = \sum_{t = 1}^n |y_t - \hat{y}_t| / \sum_{t = 1}^n |y_t|$ (доходности \%)}
	\label{fig::ford_best_metric_returns_results}
\end{figure}
Теперь, к ранее сказанному, смотрим на предсказания для тестовой выборки из которых далее делаем вывод о качестве работы обученной модели:
\begin{figure}[H]
	\centering
	\includegraphics[width=17cm]{nn/mlp/ford_test_returns_results.png}
	\caption{График реальных и предсказанных доходностей акций Ford (\%)}
	\label{fig::ford_test_returns_results}
\end{figure}
Очевидно, что полученный результат, хотя и полученная ошибка мала, не является удовлетворительным и применимым к реальности, так как, основываясь на метрике RMSE, имеем отклонение $\pm 1.99\%$, что заметно больше, чем средняя предсказанная величина. Таким образом, настоящая модель не только не может быть применена к реальности, но и способна ввести в заблуждение относительно ответа на вопрос <<Buy, Hold or sell>>. 

Также замечаем, что поведение предсказанных показателей как будто сдвинуто немного вперед. То есть в реальности уже было, а в прогнозе --- только будет. Подобное возможно по трем причинам: малая мощность модели, непригодность модели, особенность рассматриваемых данных. Значит, нельзя делать поспешные выводы о качестве самой модели, ведь анализ велся только для одной компании. Финальный ответ о пригодности той или иной модели рассматриваем в заключении настоящего исследования. Однако текущая модель включается в финальную таблицу как способная к прогнозированию как цен, так и доходностей.
\subsubsubsection{Recurrent Neural Network} \label{link::rnn_module}
\\\\
\indent Развивая тему особенностей архитектур сетей, задаемся вопросом: \textbf{Q}: MLP работает с матрицей вида объект---признак, однако как поступать в случае работы с временным рядом, ведь исходный его вид --- просто вектор? \textbf{A}: Как уже отмечалось в блоке Singular Spectrum Analysis (\myref{link::ssa}), переводим имеющийся ряд (вектор) в матричное пространство посредством применения алгоритма подобного ханкелизации. Таким образом, вместо временного ряда $y \in \R^{N}$, получаем матрицу $y \in \R^{L \times K}$, где $K  = N - L$. Для большей наглядности рассматриваем пример.
\begin{equation}
	\begin{split}
		y = \left[\begin{matrix}
			1\\2\\3\\4\\5\\6
		\end{matrix}\right], L = 4, K = 6 - 4 = 2 \Rightarrow
		X = \left[\begin{matrix}
			1 & 2\\
			2 & 3\\
			3 & 4\\
			4 & 5\\
		\end{matrix}\right]
		y = \left[\begin{matrix}
			3\\
			4\\
			5\\
			6\\
		\end{matrix}\right]
	\end{split}	
\end{equation}
Получаем то, что было нужно. Матрица объект-признак ($X$) и вектор выходов ($y$), который должен быть в итоге. Однако снова появляется вопрос \textbf{Q}: Так как работа ведется с временным рядом, целесообразно ли иметь фиксированный набор признаков, который подается на вход сети. В текущих условиях входной размерностью сети является величина $K$ (количество временных лагов). Но, что будет, если два лага найдутся не всегда? \textbf{A}: Решением этой проблемы является способ построения нейронных сетей, предложенный в \cite{hochreiter1997long, rumelhart1986learning}, представляющий из себя пошаговый сбор информации о последовательности. Данный поход получил название Recurrent Neural Network (рекуррентная нейронная сеть).

Идея в том, что на вход аналогично всему предыдущему подается наблюдение, однако, рассуждая абстрактно, имеется в виду не просто \textit{одно} наблюдение, а последовательность из них. Для лучшего понимания происходящего рассматриваем пример выше. Чтобы перевести имеющиеся $X$ и $y$ в требуемые для RNN размерности, выполняем операцию:
\begin{equation}
	\begin{split}
		X = \left[\begin{matrix}
			1 & 2\\
			2 & 3\\
			3 & 4\\
			4 & 5\\
		\end{matrix}\right]
		y = \left[\begin{matrix}
			3\\
			4\\
			5\\
			6\\
		\end{matrix}\right]
		\Rightarrow
		X = \left[\begin{matrix}
			[1], & [2]\\
			[2], & [3]\\
			[3], & [4]\\
			[4], & [5]\\
		\end{matrix}\right]
		y = \left[\begin{matrix}
			[3]\\
			[4]\\
			[5]\\
			[6]\\
		\end{matrix}\right]
	\end{split}	
\end{equation}
То есть теперь $X \in \R^{L \times K \times 1}$, а значит, приводя все в соответствие необходимой терминологии, $L$ --- размер батча, $K$ --- длина последовательности, $1$ --- количество имеющихся признаков у одного наблюдения, в нашем случае --- цена акции открытия, однако возможно наличие и других составляющих. А теперь самое интересное: подобная архитектура требует лишь фиксированного количества признаков у одного наблюдения, в то время как длина последовательности --- количество лагов, в нашем случае --- остается динамической. Следовательно, не нужно менять архитектуру сети в зависимости от количества используемых лагов --- в зависимости от длины последовательности. Данные подготовлены, теперь приступаем к систематическому изложению особенности работы алгоритма RNN.

Существует на данный момент три основных блока рекуррентных сетей: Simple Recurrent Unit \cite{rumelhart1986learning}, Long Short Term Memory (LSTM) \cite{hochreiter1997long}, Gated Recurrent Unit (GRU) \cite{cho2014learning}. Коротко говоря о каждом из них, привожу иллюстративный способ работы блоков, а также формальные вычисления.

Для начала --- Simple Recurrent Unit:
\begin{figure}[H]
	\centering
	\includegraphics[width=17cm]{nn/rnn/units/rnn_unit.png}
	\caption{Визуализация блока Simple Recurrent Unit}
	\label{fig::rnn_unit}
\end{figure}
\noindent В формальном виде, вычисления проходят следующим образом:
\begin{equation}
	\begin{split}
		h_t & = \sigma\left(W_{hh}h_{t - 1} + W_{hx} x_t + b_h\right)\\
		y_ t & = g\left(W_{yh}h_t + b_y\right)
	\end{split}
\end{equation}
Но тут сразу встает вопрос о затухании градиента при обратном проходе, то есть при применении так называемого алгоритма BackPropagation Through Time (BPTT), являющегося аналогичным классическому BackPropagation \cite{linnainmaa1970representation}. Формально данный вывод получаем посредством дифференцирования выбранной функции потерь:
\begin{equation}
	\begin{split}
		\frac{\partial L}{\partial w} = \sum_{t = 1}^T  \frac{\partial h_k}{\partial w} \left\{ \prod_{i = k + 1}^{t} \frac{\partial h_i}{\partial h_{i - 1}}\right\} \frac{\partial y_t}{\partial h_t} \frac{\partial L}{\partial y_t}
	\end{split}
\end{equation}
При этом если $\lVert \frac{\partial h_i}{\partial h_{i - 1}} \rVert_2 > 1 \Rightarrow$ взрыв, иначе $\lVert \frac{\partial h_i}{\partial h_{i - 1}} \rVert_2 < 1 \Rightarrow$ затухание в силу наличия произведения матриц Якоби. При этом --- желаемый результат --- $\lVert \frac{\partial h_i}{\partial h_{i - 1}} \rVert_2 = 1$. Решением появившейся проблемы стало изобретение в 1991 блока Long Short Term Memory, описанного в \cite{hochreiter1991untersuchungen, hochreiter1997long}. В формальном виде процесса вычислений получаем:
\begin{equation}
	\begin{split}
		f_t & = \sigma \left(W_{hh}^f h_{t - 1} + W_{hx}^f x_t + b_h^f\right)\\
		i_t & = \sigma \left(W_{hh}^i h_{t - 1} + W_{hx}^i x_t + b_h^i\right)\\
		o_t & = \sigma \left(W_{hh}^o h_{t - 1} + W_{hx}^o x_t + b_h^o\right)\\
		\tilde{c}_t & = \tanh\left(W_{hh}^c h_{t - 1} + W_{hx}^c x_t + b_c\right)\\
		c_t & = f_t c_{t - 1} + i_{t} \tilde{c}_t\\
		h_t & = o_t \tanh\left(c_t\right)
	\end{split}
\end{equation}
\noindent Графически же иллюстрация вычислительного процесс имеет более сложный вид, чем выше упомянутый Simple Recurrent Unit, однако сложность архитектуры окупается способностью данного блока избегать проблемы затухания градиента, а также --- способностью вычленять долгосрочные зависимости в имеющихся данных (в терминах сигналов --- выделение тренда).
\begin{figure}[H]
	\centering
	\includegraphics[width=17cm]{nn/rnn/units/lstm_unit.png}
	\caption{Визуализация блока Long Short Term Memory}
	\label{fig::lstm_unit}
\end{figure}
Как видим, проблема с затуханием градиента решена, однако блок получился достаточно тяжелым с точки зрения вычислений и, более того, проблема <<взрыва>> градиента никуда не делать. Вторая трудность решается посредством приема, называемого gradient clipping \cite{mikolov2012statistical}, а первая, в свою очередь, посредством далее изобретенного в 2014 блока Gated Recurrent Unit \cite{cho2014learning}, который по качеству работы сравним с LSTM, но при этом является более <<легким>> в плане количества обучаемых параметров. Формулы данного процесса:
\begin{equation}
	\begin{split}
		z_t & = \sigma\left(W_{hh}^i h_{t - 1} + W_{hx}^i x_t + b_h^z\right)\\
		r_t & = \sigma\left(W_{hh}^r h_{t - 1} + W_{hx}^r x_t + b_h^r\right)\\
		\tilde{h}_t & = \tanh \left(W_{hh}^r h_{t - 1} + W_{hx}^r x_t\right)\\
		h_t & = (1 - z_t) h_{t - 1} + z_t \tilde{h}_t
	\end{split}
\end{equation}
\noindent Визуально реализация блока GRU выглядит намного легче в плане количества параметров, чем LSTM, что позволяет предполагать увеличение скорости обучения сети, но при этом сохранение качества получаемых результатов.
\begin{figure}[H]
	\centering
	\includegraphics[width= 8cm]{nn/rnn/units/gru_unit.png}
	\caption{Визуализация блока Gated Recurrent Unit}
	\label{fig::gru_unit}
\end{figure}
Несмотря на все достоинства GRU перед LSTM невозможно однозначно сказать, какой из указанных блоков лучше. Таким образом, остается лишь один способ проверки --- эмпирически на имеющихся данных. В настоящей работе в качестве примера приводится эмпирический анализ каждого из выше указанных блоков на имеющихся данных о ценах открытия компании Ford.

Ниже приводим полученные результаты для блока Simple Recurrent Unit.
\begin{figure}[H]
	\centering
	\includegraphics[width= 17cm]{nn/rnn/block_results/rnn/ford_train_val_prices.png}
	\caption{График MSE для блока Simple Recurrent Unit (цены USD)}
	\label{fig::rnn_ford_train_val_prices}
\end{figure}
\noindent Далее смотрим на динамику лучшей валидационной метрики за все время обучения:
\begin{figure}[H]
	\centering
	\includegraphics[width= 17cm]{nn/rnn/block_results/rnn/ford_best_metric_prices.png}
	\caption{График изменения лучшей валидационной метрики: WAPE}
	\label{fig::rnn_ford_best_metric_prices}
\end{figure}
\noindent Далее --- предсказания:
\begin{figure}[H]
	\centering
	\includegraphics[width= 17cm]{nn/rnn/block_results/rnn/ford_test_prices.png}
	\caption{График реальных и предсказанных цен акций Ford (USD)}
	\label{fig::rnn_ford_test_prices}
\end{figure}
\noindent Аналогичный анализ для доходностей.
\begin{figure}[H]
	\centering
	\includegraphics[width= 17cm]{nn/rnn/block_results/rnn/ford_train_val_returns.png}
	\caption{График MSE для блока Simple Recurrent Unit (доходности \%)}
	\label{fig::rnn_ford_train_val_returns}
\end{figure}
\noindent Смотрим на динамику лучшей валидационной метрики за все время обучения:
\begin{figure}[H]
	\centering
	\includegraphics[width= 17cm]{nn/rnn/block_results/rnn/ford_best_metric_returns.png}
	\caption{График изменения лучшей валидационной метрики: WAPE}
	\label{fig::rnn_ford_best_metric_returns}
\end{figure}
\noindent Далее --- предсказания:
\begin{figure}[H]
	\centering
	\includegraphics[width= 17cm]{nn/rnn/block_results/rnn/ford_test_returns.png}
	\caption{График реальных и предсказанных доходностей Ford (\%)}
	\label{fig::rnn_ford_test_returns}
\end{figure}
\noindent Таким образом, для цен --- $0.5$ USD, что заметно меньше, чем было для MLP ($1.29$ USD), а для доходностей $2.91\%$, что больше, чем для MLP ($1.99\%$). Результат для доходностей все равно не применим к реальной жизни, так как показывает отклонение выше, чем среднее. Далее рассматриваем работу блока GRU. Аналогично: вначале цены, затем доходности.
\begin{figure}[H]
	\centering
	\includegraphics[width= 17cm]{nn/rnn/block_results/gru/ford_train_val_prices.png}
	\caption{График MSE для блока Gated Recurrent Unit (цены USD)}
	\label{fig::gru_ford_train_val_prices}
\end{figure}
\noindent Валидационная метрика за все время обучения:
\begin{figure}[H]
	\centering
	\includegraphics[width= 17cm]{nn/rnn/block_results/gru/ford_best_metric_prices.png}
	\caption{График изменения лучшей валидационной метрики: WAPE}
	\label{fig::gru_ford_best_metric_prices}
\end{figure}
\noindent Далее --- предсказания:
\begin{figure}[H]
	\centering
	\includegraphics[width= 17cm]{nn/rnn/block_results/gru/ford_test_prices.png}
	\caption{График реальных и предсказанных цен акций Ford (USD)}
	\label{fig::gru_ford_test_prices}
\end{figure}
\noindent Аналогичный анализ для доходностей.
\begin{figure}[H]
	\centering
	\includegraphics[width= 17cm]{nn/rnn/block_results/gru/ford_train_val_returns.png}
	\caption{График MSE для блока Gated Recurrent Unit (доходности \%)}
	\label{fig::gru_ford_train_val_returns}
\end{figure}
\noindent Динамика лучшей валидационной метрики:
\begin{figure}[H]
	\centering
	\includegraphics[width= 17cm]{nn/rnn/block_results/gru/ford_best_metric_returns.png}
	\caption{График изменения лучшей валидационной метрики: WAPE}
	\label{fig::gru_ford_best_metric_returns}
\end{figure}
\noindent Далее --- прогнозы:
\begin{figure}[H]
	\centering
	\includegraphics[width= 17cm]{nn/rnn/block_results/gru/ford_test_returns.png}
	\caption{График реальных и предсказанных доходностей Ford (\%)}
	\label{fig::gru_ford_test_returns}
\end{figure}
Соответственно, ошибка на ценах получилась больше на $0.05$ USD, чем у Simple Recurrent Unit ($0.5$ USD) однако ошибка на доходностях уменьшилась до $2.89\%$, что меньше на $0.1\%$ пунктов. Далее анализируем блок LSTM.
 \begin{figure}[H]
 	\centering
 	\includegraphics[width= 17cm]{nn/rnn/block_results/lstm/ford_train_val_prices.png}
 	\caption{График MSE для блока LSTM (цены USD)}
 	\label{fig::lstm_ford_train_val_prices}
 \end{figure}
 \noindent Валидационная метрика за все время обучения:
 \begin{figure}[H]
 	\centering
 	\includegraphics[width= 17cm]{nn/rnn/block_results/lstm/ford_best_metric_prices.png}
 	\caption{График изменения лучшей валидационной метрики: WAPE}
 	\label{fig::lstm_ford_best_metric_prices}
 \end{figure}
 \noindent Прогнозы:
 \begin{figure}[H]
 	\centering
 	\includegraphics[width= 17cm]{nn/rnn/block_results/lstm/ford_test_prices.png}
 	\caption{График реальных и предсказанных цен акций Ford (USD)}
 	\label{fig::lstm_ford_test_prices}
 \end{figure}
 \noindent Анализ доходностей.
 \begin{figure}[H]
 	\centering
 	\includegraphics[width= 17cm]{nn/rnn/block_results/lstm/ford_train_val_returns.png}
 	\caption{График MSE для блока LSTM (доходности \%)}
 	\label{fig::lstm_ford_train_val_returns}
 \end{figure}
 \noindent Динамика лучшей валидационной метрики:
 \begin{figure}[H]
 	\centering
 	\includegraphics[width= 17cm]{nn/rnn/block_results/lstm/ford_best_metric_returns.png}
 	\caption{График изменения лучшей валидационной метрики: WAPE}
 	\label{fig::lstm_ford_best_metric_returns}
 \end{figure}
 \noindent Предсказания:
 \begin{figure}[H]
 	\centering
 	\includegraphics[width= 17cm]{nn/rnn/block_results/lstm/ford_test_returns.png}
 	\caption{График реальных и предсказанных доходностей Ford (\%)}
 	\label{fig::lstm_ford_test_returns}
 \end{figure}
 Видим увеличение ошибки для цен на $0.07$ USD по сравнению с Simple Recurrent Unit, а также увеличение ошибки для доходностей до $2.90\%$ по сравнению с Gated Recurrent Unit ($2.89\%$).
 
 Выводом ко всему вышеописанному является то, что применение Simple Recurrent Unit для работы с ценами --- весьма неплохая идея, так как RMSE метрика уменьшилась примерно в 2 раза, однако применение какого-либо блока из класса Recurrent Network не является целесообразным для прогнозирования доходностей, тут лучше всего справляется классическая MLP модель ($1.99\%$). Однако, подводя итог, ни одна из моделей не годится для работ с доходностями в полевых условиях, так как отклонение прогнозируемого значения от реального превышает само прогнозируемое значение, таким образом, нельзя точно быть уверенными, как именно себя поведет доходность.
 
 В заключении данного блока рассмотренная модель включается в финальную сравнительную таблицу, однако точно сказать, какой именно блок необходимо использовать --- не получится, так как в среднем результаты почти одинаковые. Соответственно, основываясь на данных, проверяем валидность того или иного блока и далее --- используем его при обучении модели.
\subsubsubsection{Wavelet Network} \label{link::wavelet_nets}
\subsubsubsection{Алгоритмы обучения: AdamW vs Lions}