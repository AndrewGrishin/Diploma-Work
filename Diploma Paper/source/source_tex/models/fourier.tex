\subsubsection{Fourier analysis} \label{link::fourier_analysis}
\noindent Теперь рассматриваем второе определение стационарности (\myref{def::signal_stationarity}). То есть в настоящий момент под временным рядом понимается некоторый сигнал, полученный путем записи данных, характеризующих некоторый объект в конкретные моменты времени.

Задаемся вопросом \textbf{Q}: Можно ли как-то представить функцию вида $f(t)$ (исследуемый сигнал) в виде конечного или бесконечного наложения синусоид (под синусоидой понимается как косинус, так и синус). \textbf{A}: 1) Замечаем, что струна сама по себе при условии фиксированных концов как раз колеблется по синусоиде. Причем на ее длине всегда умещается конечное количество волн (как раз условие закрепленности струны и обеспечивает данный факт). Сама же простейшая форма колебания струны называется "гармоникой". Отсюда и название анализа: гармонический анализ. Но тогда не очевидно, можно ли представить сложную форму колебания в виде простейшей? 2) Выводим разложение в ряд Фурье и соответственно алгоритм FFT (Fast Fourier Transform).

Пусть задан некоторый периодический сигнал вида: $f: \R \to \R$ с периодом T. Тогда угадываем внешний вид разложения, а далее - формальным способом вычисляем коэффициенты в данном разложении. Нам необходимо, чтобы исходный сигнал был представим в виде:
\begin{equation} \label{equation::fourier_approximation}
	f(t) = \frac{a_0}{2} + \sum_{k = 1}^{\infty} \left\{a_k\cos(k \omega t) + b_k\sin(k \omega t)\right\}
\end{equation}
Вспоминая определение ортогональности векторов из $\R^n$, вводим по аналогии операцию ортогональности над функциями и соответственно получаем:
\begin{equation}
	p = \int_0^T g(t) \cdot f(t) \; \text{d}t
\end{equation}
Тогда в силу того, что взяты были функции $\cos(\cdot)$ и $\sin(\cdot)$, проверяем на ортогональность:
\begin{enumerate}
	\item $\sin(k\omega t), \sin(l \omega t)$.
	\begin{equation}
		p = \int_{0}^T \sin(k\omega t)\sin(l \omega t) \; \text{d}t =
		\left\{
		\begin{array}{rl}
			0 & \text{, } k \ne l\\
			\frac{T}{2} & \text{, } k = l
		\end{array}
		\right.
	\end{equation}
	
	\item $\cos(k\omega t), \cos(l \omega t)$.
	\begin{equation}
		p = \int_{0}^T \cos(k\omega t)\cos(l \omega t) \; \text{d}t =
		\left\{
		\begin{array}{rl}
			0 & \text{, } k \ne l\\
			\frac{T}{2} & \text{, } k = l
		\end{array}
		\right.
	\end{equation}

	\item $\cos(k\omega t), \sin(l \omega t)$.
	\begin{equation}
		p = \int_{0}^T \cos(k\omega t)\sin(l \omega t) \; \text{d}t = 0
	\end{equation}
\end{enumerate}
Тогда, опираясь на \ref{equation::fourier_approximation}, получаем формулы для коэффициентов:
\begin{enumerate}
	\item 
\end{enumerate}