\subsubsubsection{Пример: Multistage SSA}
\\\\
Основываясь на алгоритме, предложенном в \cite{kuang2020efficient}, и используя данные, полученные посредством функции (\ref{link::illustr_func}), получаем эффект:
\begin{figure}[H]
	\centering
	\begin{tikzpicture}
		\begin{axis}[
			grid = both,
			legend pos = north west,
			minor tick num = 1,
			major grid style = {lightgray},
			minor grid style = {lightgray!25},
			%title= {},
			width = \textwidth,
			height = 0.45 \textwidth,
			xmin=-5, xmax=5,
			ymin=-4, ymax=7.5,
			line width=0.3mm
			]
			
			\addplot[color = orange, line width = 0.035cm] table [
			x=x, 
			y=denoised, 
			col sep=comma,
			mark={},
			] {./source/source_csv/Illustration data/ssa/denoised.csv};
			
			\addplot[opacity = 0.25, color = blue] table [
			x=x, 
			y=y, 
			col sep=comma,
			mark={},
			] {./source/source_csv/Illustration data/ssa/denoised.csv};
			
			\addplot[domain = -5:5,
			samples = 300,
			color = teal,
			smooth,
			line width = 0.025cm,] {sin(deg(5 * x)) + 1 / 4 * (x^2)};
			
			\legend{$\hat{f}(x)$ - оценка $f(x)$, $f(x)$ c шумом, $f(x)$ без шума};
		\end{axis}
	\end{tikzpicture}
	\caption{Очистка ряда от шума посредством MSSA}
\end{figure}
\noindent Интересно, что в силу неоднородности дисперсии добавленного шума, сам алгоритм очистки с разной степенью качества выполняет свою работу: там, где дисперсия низкая, шум убран почти полностью (исходя из визуального анализа), там же, где дисперсия велика, выше вероятность отклонения от истинного значения функции. Следовательно, данный алгоритм не является устойчивым к рядам с гетероскедастичным шумом, значит, прежде, чем применять данный метод, необходимо провести техническое выравнивание дисперсии. Да, очистка от шума не идеальная и существенные отличия от исходного (чистого) ряда, конечно, есть, однако, сравнивая с исходными данными, видим весьма ощутимое различие. Основываясь на этом, делаем вывод, что использование алгоритма MSSA в качестве непараметризованного метода очистки ряда, позволяет назвать сам предложенный метод весьма универсальным. Под <<универсальностью>> понимается отсутствие необходимости вмешательства человека в работу программы. То есть алгоритм делает все сам без помощи извне. 

Более того, так как одной из основных задач настоящего исследования является написание программы, принимающей на вход некоторый временной ряд, а на выход дающей значение выбранной далее метрики качества для каждой из рассматриваемых моделей, универсальность метода MSSA также как и квази-универсальность (почти универсальность, закрывая глаза на малое количество гиперпараметров) позволяет использовать его в комбинации с другими алгоритмами прогнозирования, в которых конечное значение (прогноз) чувствителен к случайным шокам. Иными словами: очистка от шума необходима там, где без нее нельзя обойтись для корректного обучения модели на тренировочных данных или там, где от шума возникают серьезные ошибки при построении прогноза. При этом отмечаем, что MSSA применяется только для удаления шума, а не для формирования прогноза как такового.