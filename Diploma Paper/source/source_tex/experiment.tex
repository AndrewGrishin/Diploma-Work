\section{Описание данных и эксперимент}
После всего, описанного выше, возвращаемся к формальному изложению исследования. Получается, что для проведения сравнительного анализа используется некоторый набор моделей, включающий в себя как эконометрические подходы, так и нейросетевые. Также отмечаем, что все графики как уже представленные в настоящей работе, так и обсуждаемые далее, выполнены автором исследования на основе имеющихся данных (цена открытия, доходности), полученных с сайта <<Yahoo! Finance>>, о компаниях: Apple, Amazon, Coca Cola, Disney, EBAY, Ford Motors, General Electric, General Motors, Google, Intel, Microsoft, Netflix, Nike, Uber, Walmart --- американский рынок (New---York'ая биржа), Agricultural Bank of China, Anhui Coonch Cement, China Duty Free Group, China Life Insurance Company Limited, China Merchants Bank, China Pacific Insurance, China Shenhua, CITIC Securities, Hengrui Medicine, Kweichow Moutai, PetroChina, Ping An, SAIC Motor, Sinopec, Wanhua Chemical Group --- китайский рынок (Shanghai'ая биржа).

Для характеристики рынков были выбраны 15 компаний из разных сфер экономики. С одной стороны, такой выбор обусловлен традицией трейдеров и инвесторов хеджировать риск, диверсифицируя портфель ценных бумаг, с другой стороны, охват сфер экономики позволяет говорить о рынке в целом, а также об отсутствии побочного влияния выбранной отрасли на результат анализа. Более того для развивающихся рынков крайне характерно публичное владение среди самых крупных компаний, то есть все остальные <<мелкие игроки>> находятся все биржи. В развитых экономиках на бирже торгуются ценные бумаги как уже крупных и развитых компаний, так и пока что мелких. Таким образом, число 15 выбрано как наиболее удовлетворяющее всем требованиях как развивающихся --- выбираем только тех, что есть, так и развитых --- берем не так много, чтобы говорить именно о крупных игроках.

В целях получения репрезентативного результат в работе использовались значения цен открытия по дням с момента выхода компании на IPO до середины декабря 2022 года, когда и началось исследование. Выбранные фондовые рынки существенно отличаются возрастом --- биржи US гораздо более зрелые и, на первый взгляд, более стабильные. Китайский рынок имеет больше ограничений, чем рынок US, однако и хорошие возможности тоже. Но в сочетании со слабым управлением приводит к нестабильности и большей рискованности инвестиций (индекс волатильности высок, так как стабильно держится около $30\%$ \cite{ycharts2023ChinaETF}, а, исходя из методики расчета VIX --- volatility index --- \cite{finam2023VIX}, $30\%$ --- выше среднего, что говорит о нестабильности) в ценные бумаги, что хорошо иллюстрирует сравнительно недавно случившийся крах Шанхайской фондовой биржи, после которого биржа в целом приостановила торги на несколько дней. 

В настоящем исследовании реализуется анализ цен акций китайских компаний, котирующихся на Шанхайской бирже, соответствующей категории развивающегося рынка, в отличие от биржи в Гонконге, где торгуются иностранные ценные бумаги, которые никак не характеризуют именно локальный рынок Китая. На биржах КНР образуется несколько видов акций: A---акции, B---акции, C---акции, Red chips, P---chips, N---chips. Наибольший интерес для настоящей работы представляют акции вида A, поскольку они единственные торгуются в юанях (все остальные в Гонконгских долларах), и их (акции) могут приобрести только резиденты. Стоит отметить еще одну особенность данных акций --- до $70\%$ из них принадлежат розничным инвесторам, которые не всегда осознают риск и часто следуют за рекомендациями правительства. В Китае цена акций может сильно зависеть от внешних изменений и политики, в то время как цены акций US все-таки больше зависят от потенциала роста компаний.

В итоге получаем два рынка с далее оговоренными специфичными характеристиками, для которых, скорее всего, требуются разные математические подходы. Однако пока что это лишь догадка без доказательств. Основной метрической функцией, позволяющей делать вывод о качестве предсказательных способностей модели на 1-н рабочий день биржи и сравнивать одну модель с другой, является метрика Weighted Average Percentage Error (WAPE):
\begin{equation}
	\text{WAPE}(y, \hat{y}) = \frac{\sum_{t = 1}^n |y_t - \hat{y}_t|}{\sum_{t = 1}^n |y_t|}
\end{equation}
\noindent Где $n$ --- количество наблюдений, содержащихся в ряде, $y_t$ --- наблюдаемое значение, а $\hat{y}_t$ --- предсказанное моделью значение (оценка $y_t$). Далее рассматриваем каждый из анализируемых рынков детальнее.
\subsection{Развитый рынок (США)}
\subsection{Развивающийся рынок (Китай)}