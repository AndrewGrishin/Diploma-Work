\section{Введение} 
	Современный мир нельзя представить без валюты и денег в принципе. Каждый день всякий человек на Земле имеет с ними дело: кто-то с большей суммой, кто-то с меньшей, однако все мы неразрывно связаны с деньгами. Апогеем валютного триумфа в истории человека можно назвать создание предприятий: объединений людей на основе некоторой общей цели, которой впоследствии стали считать - получение прибыли. Логика проста: один человек получает сумму $N$ за один рабочий день $\Rightarrow$ 2 человек получат $2N$, однако, что даст им стимул к объединению усилий? Только факт того, что вместе они заработают $(2 + \varepsilon)N$ денег, где $\varepsilon$ - выигрыш от работы вместе. Но можно ли как-то получить сумму больше, чем указанная? Да, производить больше \footnote{Делай больше, клиент все купит - это очень похоже на тезис, что каждый товар находит своего покупателя $\Rightarrow$ тезис о "базаре справедливого обмена"\ Роберта Оуэна. Подробнее о том, что именно предлагалось сделать рассказано в работе \cite{ropert_ouwen}.} , пытаться подстраиваться под клиента \footnote{Человека/компании, который пользуется услугами данной компании.} : предугадывать потребности клиента и делать на конкретный товар скидку, пытаться угадать, когда спрос на тот или иной вид продукции будет больше в зависимости от времени года \footnote{Очевидно, что мороженое будут покупать заметно в меньших количествах зимой по сравнению с летом.}. То есть можно подстраиваться с двух сторон: работая над собой и изменять что-то внутри организации, а можно смотреть на то, что необходимо клиентам и делать именно это, именно тогда, когда это нужно. Получаются 2 крайности, которые необходимо уметь комбинировать. 
	
	Развивая финансовый рынок, получаем закономерное действие компаний - выход на IPO \footnote{IPO - Initial Public Offering - первый выход на биржу, когда компания продает свои акции неограниченному кругу лиц, делая их совладельцами.} и становлению их публичными. Таким образом, уже владелец компании (он же владелец акции) стремится увеличить свою выгоду от владения долей, иначе зачем ему вообще тратить деньги на покупку ценных бумаг компании. Ведь важно то, что вложенная сумма отбивалась с некоторой надбавкой, то есть человек зарабатывает то, что вложил, а также что-то сверх. Это сверх и есть искомый $\varepsilon$, который человек стремится увеличить как можно сильнее. Справедливо задать вопрос: "А зачем ему все это? Зачем вкладываться куда-то, чтобы получить больше?". Иными словами, зачем иметь больше того, чем есть сейчас? Логический ответ на это пытались дать многие экономисты, начиная с Адама Смита (1729–1737) \cite{adam_smith} и его идеей о рациональном человеке (homo economicus), однако большинство теорий, изложенных им же и позднее его последователями (членами классической школы) выдвигали это как предпосылку всего анализа. Следовательно, без выполнения данного условия о стремлении человека к максимизации собственной выгоды, большинство экономический теорий не являются работающими. 
	
	Однако, опуская этот философский вопрос и возвращаясь к методам моделирования человеческого поведения, нельзя не заметить, что сами классики не углублялись в именно математическое моделирование и как следствие - описание экономической деятельности человека посредством формул, а делали это лишь через построения причинно-следственных связей, у истоков которых стояла проблема стоимости. 
	
	Намного позже Альфред Маршалл, английский экономист, живший в 1842-1924, являясь представителем неоклассической школы (кем его "обозвал"\ Торстейн Бунде Веблен), выпускает учебник "Принципы экономической науки"\ \cite{alfred_marshall}, в котором сводит воедино все труды и знания, полученные на пути развития и становления Экономики, а также собственные работы, посвященные применению наиболее полно описанного математического подхода к анализу экономической деятельности человека. 
	
	В настоящее время для анализа экономических показателей используются всевозможные методы будь то математические или нет. Однако Математика - точная наука, значит, если получится понять, когда и сколько продукции компании будет потреблять клиент (или получать дивидендов держатель акций), то можно будет без проблем еще сильнее уменьшить издержки на производство, что в свою очередь, приведет к получение еще большей экономической прибыли. 
	
	Так как в современной науке набирает популярность применение глубоких нейронных сетей к различным видам деятельности (медицины, металлургия, космическая промышленность, киноиндустрия, материаловедение, биология, биохимия, ...) \cite{nn_all_over_us}, значит, не лишнее - проверить, а имеет ли смысл вообще применять данные методы к финансовым задачам, а конкретно к предсказанию доходностей акций. 
	
	Настоящая работа является попыткой произвести сравнительный анализ наиболее популярных на данный момент математических моделей предсказания доходностей акций, примененных к 30 компаниям развитого (США) и развивающегося (Китай: Шанхай) рынков соответственно. Сравнение производится по характеристике качества прогнозирования доходности на 1 рабочий день биржи.
	\subsection{Актуальность}
		Данное исследование актуально с двух позиций: научная - выявив наиболее удачную с точки зрения предсказания на 1 день модель, в дальнейших исследованиях можно стараться развивать только ее, чтобы получать более точные результаты, а не пытаться выбрать между тем, над какой именно моделью из их множества работать, практическая - трейдерам или просто акционерам будет намного легче воспринимать временные ряды доходностей, так как они смогут с определенной точностью предсказывать конкретное значение подобного ряда на момент времени $t + 1$.
	\subsection{Цели}
		Целью настоящего исследования является помощь трейдерам или акционерам в прогнозировании доходностей акций. А умение качественно (с определенной точностью) предсказывать доходность приводит к умению формировать ответ на вопрос вида: Buy or Sell? \footnote{Покупаем акцию или продаем?} достаточно быстро и аккуратно. Ведь основная проблема финансовых временных рядов - непредсказуемость, таким образом, нужно постараться решить данный вопрос так, чтобы на выходе было наиболее прибыльно и наименее рискованно, следовательно - наиболее точно. Отсюда появляется большая надежность финансовых инструментов с точки зрения человека, который заключает опционы или просто старается приумножить свое благосостояние посредством формирования собственного портфеля. Далее следует успешность фирмы, а отсюда, ведя разговор о финансовых рынках, возможность страны развить их еще сильнее и, возможно - увеличить благосостояние своих резидентов, а значит, возможность стать из развивающейся - развитой. 
	\subsection{Задачи}
		Задачей текущего исследования является проведение сравнительного анализа между моделями машинного и глубокого обучения с целью выявления той, которая дает наиболее точный прогноз на период длиной 1 рабочий день биржи. Для этого необходимо:
		\begin{enumerate}
			\item Загрузить данные.
			\item Предобработать данные.
			\item Провести анализ данных.
			\item Среди всех обученных алгоритмов выбрать тот, у кого лучший результат.
			\item Сформировать сравнительную таблицу между моделями по всем имеющимся данным.
		\end{enumerate}
		Несмотря на громоздкость поставленных задач, весь их комплект можно оформить в один алгоритм и, таким образом, применять его к любому набору входных данных (при условии, конечно, что входные данные - временной ряд). То есть появляется новая задача следующего вида: \textbf{на вход} программе подается временной ряд, \textbf{на выход} - показатели выбранной далее  тестовой статистики для каждой из рассматриваемых моделей. \footnote{Набор используемых моделей представлен в блоке ниже.}                  