\documentclass[a4paper, 12pt]{article}
\usepackage[english, russian]{babel}
\usepackage[left= 2cm, right= 2cm, top= 2cm, bottom= 2cm]{geometry}

\title{Раздаточный материал}
\author{Гришин~Андрей~э408}
\date{\today}

\begin{document}
	\maketitle
	\begin{enumerate}
		\item <<Buy, hold or sell>> --- вопрос о покупке, удержании или продаже по отношению к некоторому активу. Отвечая на данный вопрос, трейдер (или человек связанный с биржей) принимает решение о судьбе актива в инвестиционном портфеле.
		\item Machine Learning --- раздел искусственного интеллекта, который позволяет компьютерам <<обучаться>> на основании данных и <<приобретенного>> опыта, а не программирования.
		\item Deep Learning --- подраздел Machine Learning, который подразумевает использование глубоких нейронных сетей с множеством слоев. Позволяет компьютерам <<обучаться>> на более высоком уровне абстракции, по сравнению с классическими Machine Learning техниками.
		\item Скалограмма (Scalogram) ---  график, подставляющий собой визуализацию результата разложения имеющегося сигнала на частотные, временные и амплитудные оси.
		\item Вейвлет анализ (Wavelet Analysis) --- метод частотно--временного анализа сигналов, позволяющий разбить исследуемый сигнал на частотные компоненты. В отличие от анализа Фурье разрешает частоте сигнала изменяться во времени, что позволяет говорит о Wavelet анализе как об обобщенном анализе Фурье.
		\item EWMA (Exponentially Weighted Moving Average) --- простейшая статистическая модель, в основе которой лежит накапливание исторических знаний о временном ряде. Параметризуется коэффициентами, символизирующими степень важности исторических знаний по отношению к новым знаниям.
		\item ARIMA (Auto--Regressive Integrated Moving Average) --- классическая эконометрическая модель, основанная на корреляции между наблюдениями. Включает в себя зависимости как от предыдущих наблюдений, так и от остатков прогноза. Для достижения лучшего качества модели, описывающей временной ряд, используется методология Бокса---Дженкинса.
		\item ARFIMA (Auto--Regressive Fractionally Integrated Moving Average) --- стохастическая модель, являющаяся фрактальным расширением модели ARIMA, учитывающая дробную интегрированность ряда. Иными словами, разность между номерами наблюдений не является постоянной и/или целочисленной.
		\item GARCH (Generalized Auto--Regressive Conditional Heteroscedasticity) --- стохастическая модель, используемая для анализа временных рядов с переменной дисперсией. 
		\item FIGARCH (Fractionally Generalized Auto--Regressive Conditional Heteroscedasticity) --- стохастическая модель, являющаяся фрактальным расширением модели GARCH, учитывающая дробную интегрированность ряда. Используется для прогнозирования временных рядов с долгосрочными п\'{а}ттернами (зависимостями).
		\item SSA (Singular Spectrum Analysis) --- метод анализа временных рядов, позволяющий разложить исследуемый ряд на компоненты: тренд, сезонность, шум.
		\item MSSA (Multistage Singular Spectrum Analysis) --- расширение SSA для решения задачи удаления шума из имеющихся данных, подразумевающий итеративное разбиение сигнала на компоненты тренда и шума соответственно.
		\item MLP (Multilayer Perceptron) --- класс искусственных нейронных сетей, который состоит их нескольких слоев нейронов, каждый из которых связан с предыдущим. Данный тип сетей называется <<полносвязным>>.
		\item RNN (Recurrent Nerual Network) --- класс искусственных нейронных сетей, позволяющий обрабатывать последовательности данных. В том числе и временные ряды.
		\item WN (Wavelet Network) --- вид искусственных нейронных сетей, использующих функцию wavelet преобразования в качестве функции активации внутри сети.
		\item Трансформеры (Transformers) --- класс архитектурных нейронных сетей, применяемых для обработки последовательности данных. Наибольшее распространение подход получил в области обработки естественного языка из---за механизма самовнимания.
		\item Бустинг (Boosting) --- метод машинного обучения, позволяющий создать композитную модель из набора <<слабых>> моделей путем последовательного улучшения каждой следующей модели на основе ошибок предыдущей.
	\end{enumerate}
\end{document}