\documentclass[a4paper]{article}
\usepackage[english, russian]{babel}
\usepackage[left= 2cm, right= 3cm, top= 2cm, bottom= 2cm]{geometry}

\title{Применение Нейронных Сетей и Эконометрических Методов в прогнозировании доходностей акций}
\author{Гришин Андрей э408}

\begin{document}
	\maketitle
	
	\noindent \textbf{Область}: Финансово-техническая.\\
	
	\noindent \textbf{Тема:} Применение Нейронных Сетей и Эконометрических Методов в прогнозировании доходностей акций.\\
	
	\noindent \textbf{Актуальность}: Трейдеры смогут эффективнее (с наименьшими потерями) и оперативнее решать тактические вопросы "покупки или продажи".\\
	
	\noindent \textbf{Цель}: Показать превосходство Рекуррентных Нейронных сетей перед наиболее популярными методами прогнозирования временных рядов.\\
	
	\noindent \textbf{Описание способа достижения цели}:
		\begin{enumerate}
			\item \textbf{Конкретизация цели}: Показать наибольшую эффективность Рекуррентных Нейронных Сетей в задаче предсказания доходностей акций компании.
			\item \textbf{Метод}: Последовательное сравнение моделей на эмпирических данных развивающегося (15 компаний Китая) и развитого (15 компаний Америки) рынков.
		\end{enumerate}
	
	\noindent \textbf{Темы статей}:
		\begin{enumerate}
			\item FIGARCH \cite{figarch}
			\item SETARMA \cite{setarma}
			\item SSA \cite{ssa} \cite{ssa_2}
			\item ARFIMA \cite{arfima}
			\item Back Propagation Algorithm \cite{backpropagation}
			\item Recurrent Neural Networks \cite{rnn}
		\end{enumerate}
	
	\bibliography{cites}
	\bibliographystyle{plain}
	
\end{document}