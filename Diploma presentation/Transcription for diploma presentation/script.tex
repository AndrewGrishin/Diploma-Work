\documentclass[a4paper, 12pt]{article}
\usepackage[left= 2cm, right= 2cm, top= 2cm , bottom= 2cm]{geometry}

\title{Transcription for diploma presentation}
\author{Andrey~Grishin}
\date{\today}

\begin{document}
	\maketitle
	\section{Slide (Start)}
		Hello, my name is Andrey Grishin and today I am going to tell you about my research connected with exploring the performance of Econometric and Neural Network models on empirical data of Chinese and US markets.
	\section{Slide (About plan)}
		First of all it is crucial to mention the plan of the presentation. We will start with the introductory part that contains motivation, targets and tasks that should be completed to achieve initial goals. Then comes pre-experiment part with the foundation of my research, data analysis and insights and finally we will discuss the methods that were used to prove further established hypothesis. In conclusion, we will see the achieved statistic results up to which the question --- whether the initial hypothesis was approved or rejected --- was answered. So, let's start.
	\section{Slide (Briefly about today)}
		In today's world there is a plenty of various words in daily routine of each person. However, the most common one is "money".  It means that if we assume rationality of the individual --- here rationality means desire of a person to do everything that he/she wants and nothing that he/she does not want --- we get the aspirations to satisfy personal wishes. Hence, it can be interpreted as maximization of utility or --- in terms of money --- income.
		
		From this point the necessity of development of math models for forecasting required indicators is obvious. Due to this research, I am talking about stock market and prediction of stock returns and --- as we will see further --- prices.
		
		Initially there were developed many statistic and econometric models, but in case of the Internet development the empirical data amount grew too fast. That is why the new sphere of science, called Data Science, was developed. In contrast, nowadays there is a huge number of math models, that can be used to examine data. 
		
		So, this fact is the catalyst of competition between econometric and Machine Learning approaches to gathered data analysis. And as a subset of Machine Learning, Deep Learning is also competing with econometric and statistic approaches.
	\section{Slide (Motivation --- why)}
		According to the motivation sphere here we have 2 main branches: theoretical and practical. In case of theoretical one --- figuring out the most effective forecasting model (due to the set of them analyzed in my research) for developed and developing markets allows not to use other models and improve the only selected one. So the problem of model choice will be solved. And in practical case it will become much easier and, hence, much quicker to answer the "Buy, hold or sell" question. So the computer algorithms will be able to make much more reliable decisions that will help people to make money. As a result investors, traders and stock players are happy. 
	\section{Slide (Targets --- what we want to achieve)}
		Targets setting is a very important part of any research, that is why here are only three things, going in the following order. The main target of this work is to help traders to make more accurate decisions on "Buy, hold or sell" questions. Then, applying further mentioned algorithms user will be able to make more secure deals, hence profitable 'cause it will be possible to make Long Term forecasts. Finally, due to the number of different indicators, indexes, figures, ratios it is very difficult for not qualified user to master trading and start make money, so figuring out the best algorithm will allow this person to feel much more self-confident in investing and stop being scared of stock market. 
	\section{Slide (Tasks)}
		To be more precise, in this research I provide sequential models' comparison based on empirical dataset of prices and returns of US (15) and Chinese (15)  companies. US was taken as a developed economy, where as China (Shangahi not Hong-kong) as developing one.
	\section{Slide (Hypothesis --- what was assumed)}
		
	\section{Slide}
	\section{Slide}
	\section{Slide}
	\section{Slide}
	\section{Slide}
	\section{Slide}
	\section{Slide}
	\section{Slide}
\end{document}