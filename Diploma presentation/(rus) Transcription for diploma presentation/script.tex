\documentclass[a4paper, 12pt]{article}
\usepackage[left= 2cm, right= 2cm, top= 2cm , bottom= 2cm]{geometry}
\usepackage[english, russian]{babel}
\usepackage[utf8]{inputenc}

\title{Транскрипция к презентации}
\author{Гишин~Андрей~Юрьевич}
\date{\today}

\begin{document}
	\maketitle
	\section{Слайд (Начало)}
		Здравствуйте! Меня зовут Гришин Андрей. Я студент группы э408. И сегодня я расскажу вам о моем исследовании, связанном с изучением способности как эконометрических, так и нейросетевых моделей к прогнозированию доходностей акций в зависимости от типа анализируемого рынка. Основным объектом исследования являются доходности и цены открытия акций китайских и американских компаний.
	\section{Слайд (О плане)}
		На данном слайде для ознакомления представлен план текущей презентации.
	\section{Слайд (О <<сегодня>>)}
		 В современном мире вокруг почти каждого человека каждый день вращается множество разнообразных слов. Однако слово <<деньги>> встречается, пожалуй чаще всего. Таким образом, предполагая рациональность современного человека, получаем его желание делать то, что выгодно и не делать то, что не выгодно. В формальных терминах это можно назвать --- максимизацией индивидуальной полезности или максимизацией дохода. То есть человек хочет получать больше, чтобы мочь больше. Но, рассматривая частный случай этого, а именно --- биржу и фондовые рынки, становится понятно, что просто словами или логикой не получится точно предугадать поведение как цен, так и доходностей акций той или иной компании. Как раз для этого и придумано очень много математических подходов, однако как выбрать нужный? Вопрос сложный.
		 
		 На данный момент количество уже имеющихся в сети данных превышает огромное значение, то есть просто <<данные>> стали <<большими данными>>, для обработки которых необходимы <<большие>> модели. Следовательно, область Статистики расширилась и стала областями Искусственного Интеллекта и Эконометрики. Но там, где много подходов, высока неопределенность. 
	\section{Слайд (Актуальность --- зачем)}
		Настоящая работа призвана выбрать наилучшую среди всех рассмотренных моделей, проведя их последовательный сравнительны анализ. Выбор такой модели в будущем позволит развивать лучшее из имеющегося, а не блуждать в поисках одного из множества подходов. На практике станет проще, а значит и быстрее принимать достаточно неоднозначное биржевое решение по вопросу <<Buy, Hold or Sell>>. 
	\section{Слайд (Цели --- что хотим получить)}
		Целями настоящего исследования является помощь как трейдерам, так и кому бы то ни было, связанному с биржей, в оперативном формировании точного ответа на вопрос: <<Buy, Hold or Sell>>. Что в итоге сделает биржевые сделки менее рискованными, из чего получается планирование на более длительный промежуток времени в силу финансовой определенности. В итоге людям, например, не имеющим прямого отношения к бирже (или являющимся неопытными в этом вопросе), станет проще адаптироваться к фондовым рынкам, так как они (люди) получат возможность достаточно точно прогнозировать необходимые для принятия решения показатели.
	\section{Слайд (Задачи)}
		Задачами же моего исследования является проведение последовательного сравнительного анализа рассматриваемых эконометрических, статистических и нейросетевых моделей в задаче прогнозирования доходностей и --- как будет показано далее --- цен акций 15 американских и 15 китайских компаний сроком на 1 рабочий день биржи. Для наиболее точного результата без смещения на особенности рассматриваемой индустрии, обсуждается именно 15 компаний, взятых из разных сфер производства и услуг.
	\section{Слайд (Гипотезы --- что предполагаем)}
		Гипотезами, на которых строится все исследование, являются: Гипотеза Рыночная Эффективность Юджина Фамы, статья о которой выпущена в 1970 году, а также Гипотеза Рыночной Фрактальности, сформулированная Бенуа Мандельбротом и опубликованная им в 2006 году.
		
		В противоположность ко всем вышесказанным идет гипотеза <<рыночной неэффективности>> Мартина Севелла, опубликованная в 2011 году, по которой ГРЭ является лучшим, что может быть на сегодняшний день, однако ложным в силу наличия большого количества исследований это показывающих. В своей работе Мартин Севелл анализирует всю доступную информацию, посвященную становлению ГРЭ в том виде, в котором она нам известна.
	\section{Слайд (Больше о данных)}
		Более подробно об использованных данных. Рассматриваются компании развитого и развивающегося рынков. Но так как административный район Китая --- Гонконг --- является развитым, c точки зрения финансовой составляющей, его в анализе нет. Вместо него идет Шанхайская биржа. А для анализа американских компаний использована Нью-Йоркская биржа. Границами исследуемого временного ряда является выход компании на IPO и 13 декабря 2022 года. Так как после этой даны финансовые рынки анализируемых стран стали в полной мере подвергаться воздействию политических решений.
		
		Индустрии, анализируемые в работе включают в себя самые разнообразные отрасли производства и услуг. Некоторые из них представлены на слайде. Это IT, медиа, продажи, банки, страхование и так далее.
	\section{Слайд (Визуальный анализ данных)}
		Все графики, представленные в работе, выполнены автором при использовании таких программных пакетов как Python, MATLAB, Excel на основании собранных c сайта <<Yahoo! Finance>> данных, описанных ранее.
		
		Для лучшего понимания рассматриваемых данных, анализируем пример двух компаний: Coca Cola и Kweichow Moutai (один из крупнейших производителей алкогольной продукции Китая). Глядя на представленные графики видим 2 вещи. Первая --- на обоих верхних графиках (представлены цены) наблюдается тенденция роста с увеличением амплитуды колебаний с течением времени. Вторая --- на нижних двух графиках волатильность доходностей Coca Cola более <<плотно>> находится в пределах от -10\% до 10\%, чем аналогичный показатель для Kweichow Moutai. Отсюда вывод, что, скорее всего, потребуются разные подходы для моделирования данных показателей. Аналогично возникает чувство, что предсказывать цены проще, так как в них четко выделяется тренд и некоторый шум. Однако это пока что бездоказательный вывод. Чтобы сделать их более обоснованным, обращаемся к принципу Wavelet анализа, известного в области квантовой физики под названием <<волновой микроскоп>>. Целью данного перехода является поиск доминирующих <<частот>> в имеющемся сигнале, которые впоследствии можно использовать в прогнозировании.
	\section{Слайд (Скалограма Coca Cola)}
		Опираясь на представленный график скалограммы, имеющий по oX --- время, по oY --- частоту, а по oZ --- амплитуду (силу) частоты, имеем отсутствие какой-либо значимой частотной информации в ценах акций компании Coca Cola. Однако в доходностях --- наоборот содержатся статистически значимые показатели. Говорит ли это о целесообразности работы исключительно с доходностями?
	\section{Слайд (Скалограма Kweichow Moutai)}
		Нет! Так как на примере скалограмы Kweichow Moutai безоговорочно видны частотные паттерные, в то время как подобные паттерны доходности менее значимы. Таким образом, необходимо анализировать не что-то одно, а оба показателя. То есть цены и доходности вместе.
	\section{Слайд (Примененные методы)}
		В настоящем исследовании применяются модели как эконометрического характера, основанные на корреляции, статистические модели, вообще ничего не <<думающие>> об имеющихся данных, а также --- нейросетевые модели, держащие в своей основе сложную самообучающуюся функцию, и называемую нейронной сетью. 
		
		В эконометрической области выбраны наиболее популярные ARIMA и ее фрактальное дополнение ARFIMA, GARCH и аналогично FIGARCH.
		
		Из статистики тут EWMA (экспоненциальная средняя), а также SSA (в народе более известный как метод <<Гусеница>>), представляющий из себя сингулярное спектральное разложение имеющегося ряда.
		
		От нейросетей взят наиболее простой набор моделей (полносвязные, рекуррентные, а также wavelet сеть, заимствующая подход у Wavelet анализа), чтобы показать, как ошибка предсказаний достаточно сложных Эконометрических и Статистических моделей отличается от достаточно простых Нейросетевых. В качестве очистителя от шума имеющихся данных используется алгоритм MSSA.
		
		Наиболее популярные на данный момент в сфере обработки естественного языка <<трансформеры>> не используются. Причина --- отсутствие контекста для анализа.
		
		Топологическая функция для ранжирования полученных моделей --- WAPE (Weighted Average Percentage Error), позволяющая видеть порядок отклонения предсказанного значения от его реальной величины.
	\section{Слайд (Результат для цен)}
		Переходя к результатам, полученным для цен, видим безоговорочное превосходство нейронных сетей над Эконометрическими методами. 
		
		Однако Статистические подходы тоже дали о себе знать: EWMA делит второе место с MLP, в то время как на первом идет комбинация MLP и MSSA --- нейронная сеть на очищенных данных. Частотный анализ, представленный WN и MSSA + WN дал своей результат, однако невидный на данном графике. Рассмотрим его позже.

		Среди Эконометрических моделей, комбинация ARIMA и FIGARCH занимают первое место, что однако никак не приближает ее к состязанию с нейронными сетями или статистическими методами.
		 
		Также интересно, что, основываясь на топологической функции, Эконометрическим моделям лучше удается прогнозировать значения для развитых рынков, чем для развивающихся. Это еще одно подтверждение неполной корректности ГРЭ. Однако для нейронных сетей так сходу сделать вывод не получится в силу наличия противоположных показаний.
		
	\section{Слайд (Результат для доходностей)}
		Переходя к доходностям, видим проигрыш почти всех методов --- отклонение от реальности свыше $100\%$ у большинства моделей. Однако рекуррентные нейронные сети в сочетании с алгоритмом удаления шума показывают намного более хороший результат, что выводит их на первое место в списке всех моделей для доходностей.
		
		Столбец MLP + EWMA не вычислялся, так как проведение экспоненциального сглаживания для доходностей акций, на взгляд автора, не имеет как логического, так и экономического смысла.
		
	\section{Слайд (Особенности MLP и WN для США)}
		Рассматриваем результат, который предоставила модель WN + MSSA в сравнении с результатми MLP + MSSA. Для большинства компаний, представленных на развитом рынке США и рассмотренных в текущем исследовании, оказывается верным утверждение о наличие доминантных частотных ценовых паттернов, которые возможно использовать для процесса прогнозирования. Это подчеркивает лучшую способность предсказания у модели WN + MSSA, работающей на очищенных от шума данных, но предсказывающую исходные показатели. Однако данное утверждение верно не для всех компаний, что заставляет WN + MSSA проигрывать классическому нейросетевому подходу вида MLP + MSSA.
	
	\section{Слайд (Особенности MLP и WN для Китая)}
		В противоположность только что рассмотренному развитому рынку, для развивающегося рынка наличие доминантных частотных паттернов не является превалирующим явлением. Это видно из б\'{о}льшего количества рассмотренных компаний, для которых MLP + MSSA дает лучший результат в отличие от WN + MSSA. 
		
		Вышерассмотренные два слайда позволяют выдвинуть гипотезу о доминировании частотных паттернов на развитых рынках, что будет более подробным образом изучено в будущих работах.
	\section{Слайд (Обсуждение выводов)}
		В заключение презентации устно привожу наиболее важные выводы, полученные в результате проведения исследования, все остальные (в том числе и озвученные) представлены на слайде. Также замечание, прежде чем начну: все суждения делаются исключительно с основой на имеющихся данных и для своего уточнения могут требовать дополнительных исследований.
		
		Видим, что лучшая модели (из рассмотренного круга) для прогнозирования цен акций --- это комбинация MSSA и MLP, однако WN и MSSA иногда показывают результат лучше (как для развитых, так и для развивающихся рынков), что говорит о наличии на них доминантной частотной составляющей, используемой при прогнозировании, в то время как для доходностей --- MSSA и RNN является лучшим подходом.
		
		Очень важно, что алгоритм очистки данных от шума в среднем дает хороший прирост к точности прогноза ко всем моделям.
		
		Эконометрические методы плохо подходят как для прогнозирования цен, так и доходностей, а вот статистические (EWMA и SSA) --- показывают сравнимый с нейросетевым результат и при этом не требуют таких больших вычислительных мощности, как нейронные сети при обучении. Получаем компромисс: если нужно более точно --- нейронные сети, иначе --- статистические модели типа EWMA и SSA.
		
		Более того, алгоритм бустинга (обучение одной модели на остатках предыдущей) для Эконометрических моделей --- плохая идея. Лучше --- в случае Эконометрики --- максимизировать совокупную функцию правдоподобия.
\end{document}